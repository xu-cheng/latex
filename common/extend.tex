% Page geometry
\RequirePackage{geometry}
\geometry{
    a4paper,
    top=3cm,
    bottom=3cm,
    left=2.5cm,
    right=2.5cm
}

% Load more packages
\RequirePackage[inline]{enumitem}
\RequirePackage[compact]{titlesec}
\RequirePackage{titletoc}
\RequirePackage{tocvsec2}
\RequirePackage{fancyhdr}
\RequirePackage{afterpage}
\RequirePackage{appendix}
\RequirePackage[titles]{tocloft}
\RequirePackage{tocbibind}
\RequirePackage{footmisc}

% Variables setting
\DeclareDocumentCommand\title{m}{\gdef\@title{#1}\hypersetup{pdftitle={#1}}}
\title{}

\DeclareDocumentCommand\author{m}{\gdef\@author{#1}\hypersetup{pdfauthor={#1}}}
\author{}

% Section & TOC style
\setcounter{secnumdepth}{3}
\setcounter{tocdepth}{1}
\contentsmargin{2.0em}

<% if chapter_defined? %>
\titleformat{\chapter}{\normalfont\LARGE\bfseries\filcenter}{\chaptertitlename~\thechapter}{1em}{}
<% end %>

<% if chapter_defined? %>
\newskip\@oldcftbeforechapskip
\@oldcftbeforechapskip=\cftbeforechapskip%
<% end %>
\newskip\@oldcftbeforesecskip
\@oldcftbeforesecskip=\cftbeforesecskip%
\newskip\@oldcftbeforesubsecskip
\@oldcftbeforesubsecskip=\cftbeforesubsecskip%
<% if chapter_defined? %>
\xapptocmd{\l@chapter}{
    \ifthenelse{\value{tocdepth} > 0}{\cftbeforechapskip=0.25\baselineskip}{}
}{}{}
<% end %>
\xapptocmd{\l@section}{
<% if chapter_defined? %>
    \ifthenelse{\value{tocdepth} > 0}{\cftbeforechapskip=\@oldcftbeforechapskip}{}
<% end %>
    \ifthenelse{\value{tocdepth} > 1}{\cftbeforesecskip=0.25\baselineskip}{}
}{}{}
\xapptocmd{\l@subsection}{
    \ifthenelse{\value{tocdepth} > 1}{\cftbeforesecskip=\@oldcftbeforesecskip}{}
    \ifthenelse{\value{tocdepth} > 2}{\cftbeforesubsecskip=0.25\baselineskip}{}
}{}{}
\xapptocmd{\l@subsubsection}{
    \ifthenelse{\value{tocdepth} > 2}{\cftbeforesubsecskip=\@oldcftbeforesubsecskip}{}
}{}{}

\renewcommand*\cftfigpresnum{\figurename~}
\newlength{\@cftfignumwidth@tmp}
\settowidth{\@cftfignumwidth@tmp}{\cftfigpresnum}
\addtolength{\cftfignumwidth}{\@cftfignumwidth@tmp}
\renewcommand{\cftfigaftersnumb}{\quad~}
\renewcommand*\cfttabpresnum{\tablename~}
\newlength{\@cfttabnumwidth@tmp}
\settowidth{\@cfttabnumwidth@tmp}{\cfttabpresnum}
\addtolength{\cfttabnumwidth}{\@cfttabnumwidth@tmp}
\renewcommand{\cfttabaftersnumb}{\quad~}

% Head & Foot style
\let\ps@plain\ps@fancy
\pagestyle{fancy}
\fancyhf{}
\renewcommand{\headrulewidth}{0pt}
\renewcommand{\footrulewidth}{0pt}
\fancyfoot[C]{\thepage}

% List style
\setlist{noitemsep,partopsep=0pt,topsep=.8ex}
\setlist[1]{labelindent=\parindent}
\setlist[enumerate,1]{label=\arabic*.,ref=\arabic*}
\setlist[enumerate,2]{label*=\arabic*,ref=\theenumi.\arabic*}
\setlist[enumerate,3]{label=\emph{\alph*}),ref=\theenumii\emph{\alph*}} % chktex 9
% set enumerate* style
\BeforeBeginEnvironment{enumerate*}{%
    \setlist[enumerate,1]{label=(\roman*),ref=(\roman*)} % chktex 36
}
\AfterEndEnvironment{enumerate*}{%
    \setlist[enumerate,1]{label=\arabic*.,ref=\arabic*}
}
\setlist[description]{font=\bfseries}

% Floating objects style
% Ref: https://liam0205.me/2017/04/30/floats-in-LaTeX-the-positioning-algorithm/
\RequirePackage{float}
\setlength{\intextsep}{0.75\baselineskip plus 0.15\baselineskip minus 0.45\baselineskip} % chktex 1
\setlength{\floatsep}{0.75\baselineskip plus 0.15\baselineskip minus 0.45\baselineskip} % chktex 1
\setlength{\textfloatsep}{0.75\baselineskip plus 0.15\baselineskip minus 0.45\baselineskip} % chktex 1
\setlength{\dblfloatsep}{0.75\baselineskip plus 0.15\baselineskip minus 0.45\baselineskip} % chktex 1
\setlength{\dbltextfloatsep}{0.75\baselineskip plus 0.15\baselineskip minus 0.45\baselineskip} % chktex 1
\renewcommand{\textfraction}{0.15}
\renewcommand{\topfraction}{0.85}
\renewcommand{\dbltopfraction}{0.85}
\renewcommand{\bottomfraction}{0.65}
\renewcommand{\floatpagefraction}{0.60}

% Theorem style

\RequirePackage{amsthm}
\RequirePackage{thmtools,thm-restate}
<% if chapter_defined? %>
\declaretheorem[numberwithin=chapter,style=plain]{axiom}
\declaretheorem[numberwithin=chapter,style=definition]{definition}
\declaretheorem[numberwithin=chapter,style=definition]{example}
\declaretheorem[numberwithin=chapter,style=plain]{lemma}
\declaretheorem[numberwithin=chapter,style=plain]{theorem}
\declaretheorem[numberwithin=chapter,style=remark]{remark}
<% else %>
\declaretheorem[style=plain]{axiom}
\declaretheorem[style=definition]{definition}
\declaretheorem[style=definition]{example}
\declaretheorem[style=plain]{lemma}
\declaretheorem[style=plain]{theorem}
\declaretheorem[style=remark]{remark}
<% end %>
